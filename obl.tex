\documentclass[12pt]{article}
\usepackage{amsmath} % flere matematikkommandoer
\usepackage{amssymb}
\usepackage{mathtools}
\usepackage[utf8]{inputenc} % æøå
\usepackage[T1]{fontenc} % mere æøå
\usepackage[english]{babel} % orddeling
\usepackage{graphicx}
\usepackage{fancyvrb}
\title{Obligatorisk opgave - SS}
\author{Christian Kjær Larsen (\texttt{wkm839})\\ Lukas Svarre Engedal (\texttt{pqd294})}

\begin{document}
\maketitle

\section*{Opgave 2}

\paragraph{1.}
Området hvor funktionen $f$ er positiv er vist herunder
\begin{center}
  \includegraphics[width=0.6\textwidth]{opg21.pdf}
\end{center}
For at tjekke at $f$ er en sandsynlighedstæthed, så integreres $f$ over området $A$.
\[
    \iint_A \frac{4x^3}{y^3} dydx = \int_0^1 \left( \int_x^\infty \frac{4x^3}{y^3} dy \right) dx
    = \int_0^1 4x^3 \left[ \frac{-1}{2y^2} \right]_x^\infty dx
    = \int_0^1 2x dx = 1
\]
Det integrerer til 1 og $f$ er ikke negativ over hele $\mathbb{R}^2$. Derfor er $f$ en gyldig tæthed.
\paragraph{2.}
\begin{center}
  \includegraphics[width=0.6\textwidth]{opg22.pdf}
\end{center}
\paragraph{3.}
For at få den marginale tæthed for $X$ så integreres $y$ væk.
\[
    f_X(x)=\int_x^\infty \frac{4x^3}{y^3} dy = 2x
\]
\paragraph{4.}
\[
    E(X) = \int_0^1 xf_X(x)dx = \int_0^1 2x^2 dx = \left[ \frac{2}{3}x^3 \right]_0^1 = \frac{2}{3}
\]
\[
    E(X^2) = \int_0^1 x^2 f_X(x)dx = \int_0^1 2x^3 dx = \left[ \frac{1}{2}x^4 \right]_0^1 = \frac{1}{2}
\]
\[
    Var(X) = E(X^2) - (E(X))^2 = \frac{1}{2} - \frac{4}{9} = \frac{1}{18}
\]
\paragraph{5.}
Når $y > 1$ så kan $x$ være i intervallet $[0,1]$. Derfor er den marginale tæthed for $Y$ her
\[
    f_Y(y) = \int_0^1 \frac{4x^3}{y^3} dx = \frac{1}{y^3} \int_0^1 4x^3 dx = \frac{1}{y^3}
\]
Når $0 < y \leq 1$ så er $x$ i intervallet $[0, y]$ derfor er den marginale tæthed for $Y$ her
\[
    f_Y(y) = \int_0^y \frac{4x^3}{y^3} dx = \frac{1}{y^3} \int_0^y 4x^3 dx = y
\]
Hvis $y < 0$ så er $f_Y(y) = 0$
\paragraph{6.}
\paragraph{7.}
For at få fordelingsfunktionen, så integreres tæthedsfunktionen den marginale tæthedsfunktion for $Y$. I det andet tilfælde, når $1 < y$ så lægger vi sandsynligheden for $0 < y \leq 1$ til, ellers er det ikke en fordelingsfunktion.
\[
    G(y) =
    \begin{cases}
        \int_0^y y dy = \frac{y^2}{2} & 0<y\leq 1\\
        \int_0^1 y dy + \int_1^y y^{-3} dy = \frac{1}{2}\left[ \frac{-1}{2y^2} \right]_1^y = \frac{1}{2} + \frac{-1}{2y^2} - \frac{-1}{2} = 1-\frac{1}{2y^2} & 1 < y \\
        0 &\text{ellers}
    \end{cases}
\]
For at få den inverse, så sættes de gamle grænser ind i $G$ for at få de nye grænser. Derudover løses ligningen $y = G^{-1}(x)$ for $x$ og sættes til at være den fraktilfunktionen.
\[
    G^{-1}(y) =
    \begin{cases}
        \sqrt{2y} & 0 < y \leq \frac{1}{2} \\
        \frac{1}{\sqrt{2-2y}} & \frac{1}{2} < y \leq 1
    \end{cases}
\]
\paragraph{8.}



\end{document}
